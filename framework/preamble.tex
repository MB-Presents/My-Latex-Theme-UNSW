
% ##################################################
% Literaturverzeichnis
% ##################################################

\usepackage[style=authortitle,backend=biber]{biblatex}
\addbibresource{references.bib}

\usepackage{xpatch}
\xpatchbibmacro{cite}
  {\usebibmacro{cite:title}}
  {\usebibmacro{cite:title}%
   \setunit{\addcomma\space}%
   \usebibmacro{date}}
   {}
   {}
% ##################################################
% Tabellenverzeichnis und Tabellen
% ##################################################

\usepackage{tabularx}




% Unterdrueckung von vertikalen Linien
\usepackage{booktabs}

%Multi row für spezifische zellen
\usepackage{multirow}

%Additional table package
\usepackage{tabu}

\usepackage{longtable}

\usepackage{hhline}
\usepackage{array}
\newcolumntype{H}{>{\setbox0=\hbox\bgroup}c<{\egroup}@{}}

% ##################################################
% Definitions
% ##################################################

\newcommand{\means}{$\Rightarrow$}

% Defined Blocks

\newtheorem{remark-important}{Test}
\newtheorem{remark-info}{Thought}

\newenvironment<>{remark-important}[1][]{
    \setbeamercolor{block title example}{fg=white,bg=red!75!black}
    \begin{example} #2 [#1] } {\end{example}}
  
\newenvironment<>{remark-info}[1][]{%
  \setbeamercolor{block title example}{fg=white,bg=blue!75!black}%
  \begin{example}#2[#1]}{\end{example}}
